\documentclass[]{article}
\usepackage{lmodern}
\usepackage{amssymb,amsmath}
\usepackage{ifxetex,ifluatex}
\usepackage{fixltx2e} % provides \textsubscript
\ifnum 0\ifxetex 1\fi\ifluatex 1\fi=0 % if pdftex
  \usepackage[T1]{fontenc}
  \usepackage[utf8]{inputenc}
\else % if luatex or xelatex
  \ifxetex
    \usepackage{mathspec}
  \else
    \usepackage{fontspec}
  \fi
  \defaultfontfeatures{Ligatures=TeX,Scale=MatchLowercase}
\fi
% use upquote if available, for straight quotes in verbatim environments
\IfFileExists{upquote.sty}{\usepackage{upquote}}{}
% use microtype if available
\IfFileExists{microtype.sty}{%
\usepackage{microtype}
\UseMicrotypeSet[protrusion]{basicmath} % disable protrusion for tt fonts
}{}
\usepackage[margin=1in]{geometry}
\usepackage{hyperref}
\hypersetup{unicode=true,
            pdftitle={Single studies using the CaseCrossover package},
            pdfauthor={Martijn J. Schuemie},
            pdfborder={0 0 0},
            breaklinks=true}
\urlstyle{same}  % don't use monospace font for urls
\usepackage{color}
\usepackage{fancyvrb}
\newcommand{\VerbBar}{|}
\newcommand{\VERB}{\Verb[commandchars=\\\{\}]}
\DefineVerbatimEnvironment{Highlighting}{Verbatim}{commandchars=\\\{\}}
% Add ',fontsize=\small' for more characters per line
\usepackage{framed}
\definecolor{shadecolor}{RGB}{248,248,248}
\newenvironment{Shaded}{\begin{snugshade}}{\end{snugshade}}
\newcommand{\AlertTok}[1]{\textcolor[rgb]{0.94,0.16,0.16}{#1}}
\newcommand{\AnnotationTok}[1]{\textcolor[rgb]{0.56,0.35,0.01}{\textbf{\textit{#1}}}}
\newcommand{\AttributeTok}[1]{\textcolor[rgb]{0.77,0.63,0.00}{#1}}
\newcommand{\BaseNTok}[1]{\textcolor[rgb]{0.00,0.00,0.81}{#1}}
\newcommand{\BuiltInTok}[1]{#1}
\newcommand{\CharTok}[1]{\textcolor[rgb]{0.31,0.60,0.02}{#1}}
\newcommand{\CommentTok}[1]{\textcolor[rgb]{0.56,0.35,0.01}{\textit{#1}}}
\newcommand{\CommentVarTok}[1]{\textcolor[rgb]{0.56,0.35,0.01}{\textbf{\textit{#1}}}}
\newcommand{\ConstantTok}[1]{\textcolor[rgb]{0.00,0.00,0.00}{#1}}
\newcommand{\ControlFlowTok}[1]{\textcolor[rgb]{0.13,0.29,0.53}{\textbf{#1}}}
\newcommand{\DataTypeTok}[1]{\textcolor[rgb]{0.13,0.29,0.53}{#1}}
\newcommand{\DecValTok}[1]{\textcolor[rgb]{0.00,0.00,0.81}{#1}}
\newcommand{\DocumentationTok}[1]{\textcolor[rgb]{0.56,0.35,0.01}{\textbf{\textit{#1}}}}
\newcommand{\ErrorTok}[1]{\textcolor[rgb]{0.64,0.00,0.00}{\textbf{#1}}}
\newcommand{\ExtensionTok}[1]{#1}
\newcommand{\FloatTok}[1]{\textcolor[rgb]{0.00,0.00,0.81}{#1}}
\newcommand{\FunctionTok}[1]{\textcolor[rgb]{0.00,0.00,0.00}{#1}}
\newcommand{\ImportTok}[1]{#1}
\newcommand{\InformationTok}[1]{\textcolor[rgb]{0.56,0.35,0.01}{\textbf{\textit{#1}}}}
\newcommand{\KeywordTok}[1]{\textcolor[rgb]{0.13,0.29,0.53}{\textbf{#1}}}
\newcommand{\NormalTok}[1]{#1}
\newcommand{\OperatorTok}[1]{\textcolor[rgb]{0.81,0.36,0.00}{\textbf{#1}}}
\newcommand{\OtherTok}[1]{\textcolor[rgb]{0.56,0.35,0.01}{#1}}
\newcommand{\PreprocessorTok}[1]{\textcolor[rgb]{0.56,0.35,0.01}{\textit{#1}}}
\newcommand{\RegionMarkerTok}[1]{#1}
\newcommand{\SpecialCharTok}[1]{\textcolor[rgb]{0.00,0.00,0.00}{#1}}
\newcommand{\SpecialStringTok}[1]{\textcolor[rgb]{0.31,0.60,0.02}{#1}}
\newcommand{\StringTok}[1]{\textcolor[rgb]{0.31,0.60,0.02}{#1}}
\newcommand{\VariableTok}[1]{\textcolor[rgb]{0.00,0.00,0.00}{#1}}
\newcommand{\VerbatimStringTok}[1]{\textcolor[rgb]{0.31,0.60,0.02}{#1}}
\newcommand{\WarningTok}[1]{\textcolor[rgb]{0.56,0.35,0.01}{\textbf{\textit{#1}}}}
\usepackage{graphicx,grffile}
\makeatletter
\def\maxwidth{\ifdim\Gin@nat@width>\linewidth\linewidth\else\Gin@nat@width\fi}
\def\maxheight{\ifdim\Gin@nat@height>\textheight\textheight\else\Gin@nat@height\fi}
\makeatother
% Scale images if necessary, so that they will not overflow the page
% margins by default, and it is still possible to overwrite the defaults
% using explicit options in \includegraphics[width, height, ...]{}
\setkeys{Gin}{width=\maxwidth,height=\maxheight,keepaspectratio}
\IfFileExists{parskip.sty}{%
\usepackage{parskip}
}{% else
\setlength{\parindent}{0pt}
\setlength{\parskip}{6pt plus 2pt minus 1pt}
}
\setlength{\emergencystretch}{3em}  % prevent overfull lines
\providecommand{\tightlist}{%
  \setlength{\itemsep}{0pt}\setlength{\parskip}{0pt}}
\setcounter{secnumdepth}{5}
% Redefines (sub)paragraphs to behave more like sections
\ifx\paragraph\undefined\else
\let\oldparagraph\paragraph
\renewcommand{\paragraph}[1]{\oldparagraph{#1}\mbox{}}
\fi
\ifx\subparagraph\undefined\else
\let\oldsubparagraph\subparagraph
\renewcommand{\subparagraph}[1]{\oldsubparagraph{#1}\mbox{}}
\fi

%%% Use protect on footnotes to avoid problems with footnotes in titles
\let\rmarkdownfootnote\footnote%
\def\footnote{\protect\rmarkdownfootnote}

%%% Change title format to be more compact
\usepackage{titling}

% Create subtitle command for use in maketitle
\newcommand{\subtitle}[1]{
  \posttitle{
    \begin{center}\large#1\end{center}
    }
}

\setlength{\droptitle}{-2em}

  \title{Single studies using the CaseCrossover package}
    \pretitle{\vspace{\droptitle}\centering\huge}
  \posttitle{\par}
    \author{Martijn J. Schuemie}
    \preauthor{\centering\large\emph}
  \postauthor{\par}
      \predate{\centering\large\emph}
  \postdate{\par}
    \date{2018-11-23}


\begin{document}
\maketitle

{
\setcounter{tocdepth}{2}
\tableofcontents
}
\hypertarget{introduction}{%
\section{Introduction}\label{introduction}}

This vignette describes how you can use the \texttt{CaseCrossover}
package to perform a single case-crossover study. We will walk through
all the steps needed to perform an exemplar study, and we have selected
the well-studied topic of the effect of NSAIDs on gastrointestinal (GI)
bleeding-related hospitalization. For simplicity, we focus on one NSAID:
diclofenac.

\hypertarget{installation-instructions}{%
\section{Installation instructions}\label{installation-instructions}}

Before installing the \texttt{CaseCrossover} package make sure you have
Java available. Java can be downloaded from
\href{http://www.java.com}{www.java.com}. For Windows users, RTools is
also necessary. RTools can be downloaded from
\href{http://cran.r-project.org/bin/windows/Rtools/}{CRAN}.

The \texttt{CaseCrossover} package is currently maintained in a
\href{https://github.com/OHDSI/CaseCrossover}{Github repository}, and
has dependencies on other packages in Github. All of these packages can
be downloaded and installed from within R using the \texttt{drat}
package:

\begin{Shaded}
\begin{Highlighting}[]
\KeywordTok{install.packages}\NormalTok{(}\StringTok{"drat"}\NormalTok{)}
\NormalTok{drat}\OperatorTok{::}\KeywordTok{addRepo}\NormalTok{(}\StringTok{"OHDSI"}\NormalTok{)}
\KeywordTok{install.packages}\NormalTok{(}\StringTok{"CaseCrossover"}\NormalTok{)}
\end{Highlighting}
\end{Shaded}

Once installed, you can type \texttt{library(CaseCrossover)} to load the
package.

\hypertarget{overview}{%
\section{Overview}\label{overview}}

In the \texttt{CaseCrossover} package a study requires four steps:

\begin{enumerate}
\def\labelenumi{\arabic{enumi}.}
\tightlist
\item
  Loading data on the cases (and potential controls when performing a
  case-time-control analysis) from the database needed for matching.
\item
  Selecting subjects to include in the study.
\item
  Determining exposure status for cases (and controls) based on a
  definition of the risk windows.
\item
  Fitting the model using conditional logistic regression.
\end{enumerate}

In the following sections these steps will be demonstrated.

\hypertarget{configuring-the-connection-to-the-server}{%
\section{Configuring the connection to the
server}\label{configuring-the-connection-to-the-server}}

We need to tell R how to connect to the server where the data are.
\texttt{CaseCrossover} uses the \texttt{DatabaseConnector} package,
which provides the \texttt{createConnectionDetails} function. Type
\texttt{?createConnectionDetails} for the specific settings required for
the various database management systems (DBMS). For example, one might
connect to a PostgreSQL database using this code:

\begin{Shaded}
\begin{Highlighting}[]
\NormalTok{connectionDetails <-}\StringTok{ }\KeywordTok{createConnectionDetails}\NormalTok{(}\DataTypeTok{dbms =} \StringTok{"postgresql"}\NormalTok{, }
                                             \DataTypeTok{server =} \StringTok{"localhost/ohdsi"}\NormalTok{, }
                                             \DataTypeTok{user =} \StringTok{"joe"}\NormalTok{, }
                                             \DataTypeTok{password =} \StringTok{"supersecret"}\NormalTok{)}

\NormalTok{cdmDatabaseSchema <-}\StringTok{ "my_cdm_data"}
\NormalTok{cohortDatabaseSchema <-}\StringTok{ "my_results"}
\NormalTok{cohortTable <-}\StringTok{ "my_cohorts"}
\NormalTok{cdmVersion <-}\StringTok{ "5"}
\end{Highlighting}
\end{Shaded}

The last three lines define the \texttt{cdmDatabaseSchema} and
\texttt{cohortDatabaseSchema} variables,as well as the CDM version.
We'll use these later to tell R where the data in CDM format live, where
we have stored our cohorts of interest, and what version CDM is used.
Note that for Microsoft SQL Server, databaseschemas need to specify both
the database and the schema, so for example
\texttt{cdmDatabaseSchema\ \textless{}-\ "my\_cdm\_data.dbo"}.

\hypertarget{preparing-the-health-outcome-of-interest-and-nesting-cohort}{%
\section{Preparing the health outcome of interest and nesting
cohort}\label{preparing-the-health-outcome-of-interest-and-nesting-cohort}}

We need to define the exposures and outcomes for our study.
Additionally, we can specify a cohort in which to nest the study. The
CDM also already contains standard cohorts in the \texttt{drug\_era} and
\texttt{condition\_era} table that could be used if those meet the
requirement of the study, but often we require custom cohort
definitions. One way to define cohorts is by writing SQL statements
against the OMOP CDM that populate a table of events in which we are
interested. The resulting table should have the same structure as the
\texttt{cohort} table in the CDM, meaning it should have the fields
\texttt{cohort\_definition\_id}, \texttt{cohort\_start\_date},
\texttt{cohort\_end\_date},and \texttt{subject\_id}.

For our example study, we will rely on \texttt{drug\_era} to define
exposures, and we have created a file called \emph{vignette.sql} with
the following contents to define the outcome and the nesting cohort:

\begin{Shaded}
\begin{Highlighting}[]
\CommentTok{/***********************************}
\CommentTok{File vignette.sql }
\CommentTok{***********************************/}

\KeywordTok{IF}\NormalTok{ OBJECT_ID(}\StringTok{'@cohortDatabaseSchema.@cohortTable'}\NormalTok{, }\StringTok{'U'}\NormalTok{) }\KeywordTok{IS} \KeywordTok{NOT} \KeywordTok{NULL}
  \KeywordTok{DROP} \KeywordTok{TABLE}\NormalTok{ @cohortDatabaseSchema.@cohortTable;}

\KeywordTok{SELECT} \DecValTok{1} \KeywordTok{AS}\NormalTok{ cohort_definition_id,}
\NormalTok{    condition_start_date }\KeywordTok{AS}\NormalTok{ cohort_start_date,}
\NormalTok{    condition_end_date }\KeywordTok{AS}\NormalTok{ cohort_end_date,}
\NormalTok{    condition_occurrence.person_id }\KeywordTok{AS}\NormalTok{ subject_id}
\KeywordTok{INTO}\NormalTok{ @cohortDatabaseSchema.@cohortTable}
\KeywordTok{FROM}\NormalTok{ @cdmDatabaseSchema.condition_occurrence}
\KeywordTok{INNER} \KeywordTok{JOIN}\NormalTok{ @cdmDatabaseSchema.visit_occurrence}
    \KeywordTok{ON}\NormalTok{ condition_occurrence.visit_occurrence_id = visit_occurrence.visit_occurrence_id}
\KeywordTok{WHERE}\NormalTok{ condition_concept_id }\KeywordTok{IN}\NormalTok{ (}
        \KeywordTok{SELECT}\NormalTok{ descendant_concept_id}
        \KeywordTok{FROM}\NormalTok{ @cdmDatabaseSchema.concept_ancestor}
        \KeywordTok{WHERE}\NormalTok{ ancestor_concept_id = }\DecValTok{192671} \CommentTok{-- GI - Gastrointestinal haemorrhage}
\NormalTok{        )}
    \KeywordTok{AND}\NormalTok{ visit_occurrence.visit_concept_id }\KeywordTok{IN}\NormalTok{ (}\DecValTok{9201}\NormalTok{, }\DecValTok{9203}\NormalTok{);}
    
\KeywordTok{INSERT} \KeywordTok{INTO}\NormalTok{ @cohortDatabaseSchema.@cohortTable }
\NormalTok{(cohort_definition_id, cohort_start_date, cohort_end_date, subject_id)}
\KeywordTok{SELECT} \DecValTok{2} \KeywordTok{AS}\NormalTok{ cohort_definition_id,}
    \FunctionTok{MIN}\NormalTok{(condition_start_date) }\KeywordTok{AS}\NormalTok{ cohort_start_date,}
    \KeywordTok{NULL} \KeywordTok{AS}\NormalTok{ cohort_end_date,}
\NormalTok{    person_id }\KeywordTok{AS}\NormalTok{ subject_id}
\KeywordTok{FROM}\NormalTok{ @cdmDatabaseSchema.condition_occurrence}
\KeywordTok{WHERE}\NormalTok{ condition_concept_id }\KeywordTok{IN}\NormalTok{ (}
        \KeywordTok{SELECT}\NormalTok{ descendant_concept_id}
        \KeywordTok{FROM}\NormalTok{ @cdmDatabaseSchema.concept_ancestor}
        \KeywordTok{WHERE}\NormalTok{ ancestor_concept_id = }\DecValTok{80809} \CommentTok{-- rheumatoid arthritis}
\NormalTok{        )}
\KeywordTok{GROUP} \KeywordTok{BY}\NormalTok{ person_id;}
\end{Highlighting}
\end{Shaded}

This is parameterized SQL which can be used by the \texttt{SqlRender}
package. We use parameterized SQL so we do not have to pre-specify the
names of the CDM and cohort schemas. That way, if we want to run the SQL
on a different schema, we only need to change the parameter values; we
do not have to change the SQL code. By also making use of translation
functionality in \texttt{SqlRender}, we can make sure the SQL code can
be run in many different environments.

\begin{Shaded}
\begin{Highlighting}[]
\KeywordTok{library}\NormalTok{(SqlRender)}
\NormalTok{sql <-}\StringTok{ }\KeywordTok{readSql}\NormalTok{(}\StringTok{"vignette.sql"}\NormalTok{)}
\NormalTok{sql <-}\StringTok{ }\KeywordTok{renderSql}\NormalTok{(sql,}
                 \DataTypeTok{cdmDatabaseSchema =}\NormalTok{ cdmDatabaseSchema, }
                 \DataTypeTok{cohortDatabaseSchema =}\NormalTok{ cohortDatabaseSchema}
                 \DataTypeTok{cohortTable =}\NormalTok{ cohortTable)}\OperatorTok{$}\NormalTok{sql}
\NormalTok{sql <-}\StringTok{ }\KeywordTok{translateSql}\NormalTok{(sql, }\DataTypeTok{targetDialect =}\NormalTok{ connectionDetails}\OperatorTok{$}\NormalTok{dbms)}\OperatorTok{$}\NormalTok{sql}

\NormalTok{connection <-}\StringTok{ }\KeywordTok{connect}\NormalTok{(connectionDetails)}
\KeywordTok{executeSql}\NormalTok{(connection, sql)}
\end{Highlighting}
\end{Shaded}

In this code, we first read the SQL from the file into memory. In the
next line, we replace the three parameter names with the actual values.
We then translate the SQL into the dialect appropriate for the DBMS we
already specified in the \texttt{connectionDetails}. Next, we connect to
the server, and submit the rendered and translated SQL.

If all went well, we now have a table with the outcome of interest and
the nesting cohort. We can see how many events:

\begin{Shaded}
\begin{Highlighting}[]
\NormalTok{sql <-}\StringTok{ }\KeywordTok{paste}\NormalTok{(}\StringTok{"SELECT cohort_definition_id, COUNT(*) AS count"}\NormalTok{,}
             \StringTok{"FROM @cohortDatabaseSchema.@cohortTable"}\NormalTok{,}
             \StringTok{"GROUP BY cohort_definition_id"}\NormalTok{)}
\NormalTok{sql <-}\StringTok{ }\KeywordTok{renderSql}\NormalTok{(sql, }
                 \DataTypeTok{cohortDatabaseSchema =}\NormalTok{ cohortDatabaseSchema, }
                 \DataTypeTok{cohortTable =}\NormalTok{ cohortTable)}\OperatorTok{$}\NormalTok{sql}
\NormalTok{sql <-}\StringTok{ }\KeywordTok{translateSql}\NormalTok{(sql, }\DataTypeTok{targetDialect =}\NormalTok{ connectionDetails}\OperatorTok{$}\NormalTok{dbms)}\OperatorTok{$}\NormalTok{sql}

\KeywordTok{querySql}\NormalTok{(connection, sql)}
\end{Highlighting}
\end{Shaded}

\begin{verbatim}
#>   cohort_definition_id  count
#> 1                    1 422274
#> 2                    2 118430
\end{verbatim}

\hypertarget{extracting-the-data-from-the-server}{%
\section{Extracting the data from the
server}\label{extracting-the-data-from-the-server}}

Now we can tell \texttt{CaseCrossover} to extract the necessary data on
the cases:

\begin{Shaded}
\begin{Highlighting}[]
\NormalTok{caseCrossoverData <-}\StringTok{ }\KeywordTok{getDbCaseCrossoverData}\NormalTok{(}\DataTypeTok{connectionDetails =}\NormalTok{ connectionDetails,}
                                            \DataTypeTok{cdmDatabaseSchema =}\NormalTok{ cdmDatabaseSchema,}
                                            \DataTypeTok{oracleTempSchema =}\NormalTok{ oracleTempSchema,}
                                            \DataTypeTok{outcomeDatabaseSchema =}\NormalTok{ cohortDatabaseSchema,}
                                            \DataTypeTok{outcomeTable =}\NormalTok{ cohortTable,}
                                            \DataTypeTok{outcomeId =} \DecValTok{1}\NormalTok{,}
                                            \DataTypeTok{exposureDatabaseSchema =}\NormalTok{ cdmDatabaseSchema,}
                                            \DataTypeTok{exposureTable =} \StringTok{"drug_era"}\NormalTok{,}
                                            \DataTypeTok{exposureIds =} \DecValTok{1124300}\NormalTok{,}
                                            \DataTypeTok{useNestingCohort =} \OtherTok{TRUE}\NormalTok{,}
                                            \DataTypeTok{nestingCohortDatabaseSchema =}\NormalTok{ cohortDatabaseSchema,}
                                            \DataTypeTok{nestingCohortTable =}\NormalTok{ cohortTable,}
                                            \DataTypeTok{nestingCohortId =} \DecValTok{2}\NormalTok{,}
                                            \DataTypeTok{useObservationEndAsNestingEndDate =} \OtherTok{TRUE}\NormalTok{,}
                                            \DataTypeTok{getTimeControlData =} \OtherTok{TRUE}\NormalTok{)}


\NormalTok{caseCrossoverData}
\end{Highlighting}
\end{Shaded}

\begin{verbatim}
#> Case-crossover data object
#> 
#> Outcome concept ID(s): 1
#> Nesting cohort ID: 2
#> Exposure concept ID(s): 1124300
\end{verbatim}

There are many parameters, but they are all documented in the
\texttt{CaseCrossover} manual. In short, we are pointing the function to
the table created earlier and indicating which concept ID in that table
identifies the outcome. Note that it is possible to fetch the data for
multiple outcomes at once for efficiency. We specify that we will use
the \texttt{drug\_era} table to identify exposures, and will only
retrieve data on exposure to Diclofenac (concept ID 1124300). We
furthermore specify a nesting cohort in the same table, meaning that
people will be eligible to be cases if and when they fall inside the
specified cohort. In this case, the nesting cohort starts when people
have their first diagnosis of rheumatoid arthritis. We use the
\texttt{useObservationEndAsNestingEndDate} argument to indicate people
will stay eligible until the end of their observation period. We
furthermore specify we want to retrieve data on time controls, which
will be used later to adjust for time-trends in exposures, effectively
turning the case-crossover study into a case-time-control study.

Data about the cases (and potential time controls) are extracted from
the server and stored in the \texttt{caseCrossoverData} object. This
object uses the package \texttt{ff} to store information in a way that
ensures R does not run out of memory, even when the data are large.

We can use the generic \texttt{summary()} function to view some more
information of the data we extracted:

\begin{Shaded}
\begin{Highlighting}[]
\KeywordTok{summary}\NormalTok{(caseCrossoverData)}
\end{Highlighting}
\end{Shaded}

\begin{verbatim}
#> Case-crossover data object summary
#> 
#> Outcome concept ID(s): 1
#> Nesting cohort ID: 2
#> 
#> Population count: 168765
#> Population window count: 168765
#> 
#> Outcome counts:
#>   Event count Case count
#> 1       20309      12821
#> 
#> Exposure counts:
#>         Exposure count Person count
#> 1124300          43721        21977
\end{verbatim}

\hypertarget{saving-the-data-to-file}{%
\subsection{Saving the data to file}\label{saving-the-data-to-file}}

Creating the \texttt{caseCrossoverData} object can take considerable
computing time, and it is probably a good idea to save it for future
sessions. Because \texttt{caseCrossoverData} uses \texttt{ff}, we cannot
use R's regular save function. Instead, we'll have to use the
\texttt{saveCaseCrossoverData()} function:

\begin{Shaded}
\begin{Highlighting}[]
\KeywordTok{saveCaseCrossoverData}\NormalTok{(caseCrossoverData, }\StringTok{"GiBleed"}\NormalTok{)}
\end{Highlighting}
\end{Shaded}

We can use the \texttt{loadCaseCrossoverData()} function to load the
data in a future session.

\hypertarget{selecting-subjects}{%
\section{Selecting subjects}\label{selecting-subjects}}

Next, we can use the data to select matched controls per case:

\begin{Shaded}
\begin{Highlighting}[]
\NormalTok{subjects <-}\StringTok{ }\KeywordTok{selectSubjectsToInclude}\NormalTok{(}\DataTypeTok{caseCrossoverData =}\NormalTok{ caseCrossoverData,}
                                    \DataTypeTok{outcomeId =} \DecValTok{1}\NormalTok{,}
                                    \DataTypeTok{firstOutcomeOnly =} \OtherTok{TRUE}\NormalTok{,}
                                    \DataTypeTok{washoutPeriod =} \DecValTok{183}\NormalTok{)}
\end{Highlighting}
\end{Shaded}

In this example, we specify a washout period of 180 days, meaning that
cases (and controls) are required to have a minimum of 180 days of
observation prior to the index date. We also specify we will only
consider the first outcome per person. If a person's first outcome is
within the washout period, that person will be removed from the
analysis.

The \texttt{subjects} object is a data frame with five columns:

\begin{Shaded}
\begin{Highlighting}[]
\KeywordTok{head}\NormalTok{(subjects)}
\end{Highlighting}
\end{Shaded}

\begin{verbatim}
#>   personId  indexDate isCase stratumId observationPeriodStartDate
#> 1        3 2009-10-10   TRUE         1                 2001-10-12
#> 2      123 2009-10-11   TRUE         2                 2002-01-11
#> 3      345 2009-10-09   TRUE         3                 2001-05-03
#> 4        6 2010-05-04   TRUE         4                 2003-02-01
#> 5      234 2010-05-04   TRUE         5                 2007-01-01
#> 6      567 2010-05-05   TRUE         6                 2006-03-01
\end{verbatim}

We can show the attrition to see why cases and events were filtered:

\begin{Shaded}
\begin{Highlighting}[]
\KeywordTok{getAttritionTable}\NormalTok{(subjects)}
\end{Highlighting}
\end{Shaded}

\begin{verbatim}
#>                      description eventCount caseCount
#> 1                Original counts      20309     12821
#> 2               First event only      12821     12821
#> 3 Require 183 days of prior obs.       7555      7555
\end{verbatim}

\hypertarget{determining-exposure-status}{%
\section{Determining exposure
status}\label{determining-exposure-status}}

We can now evaluate the exposure status of the cases in various time
windows relative to the index date:

\begin{Shaded}
\begin{Highlighting}[]
\NormalTok{exposureStatus <-}\StringTok{ }\KeywordTok{getExposureStatus}\NormalTok{(}\DataTypeTok{subjects =}\NormalTok{ subjects,}
                                    \DataTypeTok{caseCrossoverData =}\NormalTok{ caseCrossoverData,}
                                    \DataTypeTok{exposureId =} \DecValTok{1124300}\NormalTok{,}
                                    \DataTypeTok{firstExposureOnly =} \OtherTok{FALSE}\NormalTok{,}
                                    \DataTypeTok{riskWindowStart =} \DecValTok{-30}\NormalTok{,}
                                    \DataTypeTok{riskWindowEnd =} \DecValTok{0}\NormalTok{,}
                                    \DataTypeTok{controlWindowOffsets =} \KeywordTok{c}\NormalTok{(}\OperatorTok{-}\DecValTok{60}\NormalTok{))}
\end{Highlighting}
\end{Shaded}

Here we specify we are intested in all exposures, not just the first
one, and that we will use two windows per subject: a case window defined
as the 30 days preceding (and including) the index date, and a control
window which has the same length as the case window but is shifted 60
days backwards, so from 90 days to (and including) 60 days prior to the
index date. Note that multiple control windows can be specified by
specifying more control window offsets.

Exposure status is then determined based on whether an exposure overlaps
with one of the windows. The resulting \texttt{exposureStatus} object is
a data frame with six columns:

\begin{Shaded}
\begin{Highlighting}[]
\KeywordTok{head}\NormalTok{(exposureStatus)}
\end{Highlighting}
\end{Shaded}

\begin{verbatim}
#>   personId  indexDate isCase stratumId isCaseWindow exposed
#> 1        3 2009-10-10   TRUE         1         TRUE       0
#> 2      123 2009-10-11   TRUE         2         TRUE       0
#> 3      345 2009-10-09   TRUE         3         TRUE       0
#> 4        6 2010-05-04   TRUE         4         TRUE       0
#> 5      234 2010-05-04   TRUE         5         TRUE       0
#> 6      567 2010-05-05   TRUE         6         TRUE       0
\end{verbatim}

\hypertarget{fitting-the-model}{%
\section{Fitting the model}\label{fitting-the-model}}

We can now fit the model, which is a logistic regression conditioned on
the matched sets:

\begin{Shaded}
\begin{Highlighting}[]
\NormalTok{fit <-}\StringTok{ }\KeywordTok{fitCaseCrossoverModel}\NormalTok{(exposureStatus)}

\NormalTok{fit}
\end{Highlighting}
\end{Shaded}

\begin{verbatim}
#> Case-Crossover fitted model
#> Status: OK
#> 
#>           Estimate lower .95 upper .95    logRr seLogRr
#> treatment 1.068966  0.747257  1.529176 0.066691  0.1827
\end{verbatim}

The generic functions \texttt{summary}, \texttt{coef}, and
\texttt{confint} are implemented for the \texttt{fit} object:

\begin{Shaded}
\begin{Highlighting}[]
\KeywordTok{summary}\NormalTok{(fit)}
\end{Highlighting}
\end{Shaded}

\begin{verbatim}
#> Case-Crossover fitted model
#> Status: OK
#> 
#>           Estimate lower .95 upper .95    logRr seLogRr
#> treatment 1.068966  0.747257  1.529176 0.066691  0.1827
#> 
#> Counts
#>       Cases Controls Control win. (cases) Control win. (controls)
#> Count  7555        0                 7555                       0
#>       Exposed case win. (cases) Exposed control win. (cases)
#> Count                       160                          156
#>       Exposed case win. (controls) Exposed control win. (controls)
#> Count                            0                               0
\end{verbatim}

\begin{Shaded}
\begin{Highlighting}[]
\KeywordTok{coef}\NormalTok{(fit)}
\end{Highlighting}
\end{Shaded}

\begin{verbatim}
#> [1] 0.06669137
\end{verbatim}

\begin{Shaded}
\begin{Highlighting}[]
\KeywordTok{confint}\NormalTok{(fit)}
\end{Highlighting}
\end{Shaded}

\begin{verbatim}
#> [1] -0.2913464  0.4247292
\end{verbatim}

\hypertarget{case-time-control}{%
\section{Case-time-control}\label{case-time-control}}

A variant of the case-crossover design is the case-time-control design.
This design adjusts for time-trends in exposure by using a set of
control subjects. To use this design in the \texttt{CaseCrossover}
package, one needs to simply provide matching criteria to the
\texttt{selectSubjectsToInclude} function:

\begin{Shaded}
\begin{Highlighting}[]
\NormalTok{matchingCriteria <-}\StringTok{ }\KeywordTok{createMatchingCriteria}\NormalTok{(}\DataTypeTok{controlsPerCase =} \DecValTok{1}\NormalTok{,}
                                           \DataTypeTok{matchOnAge =} \OtherTok{TRUE}\NormalTok{,}
                                           \DataTypeTok{ageCaliper =} \DecValTok{2}\NormalTok{,}
                                           \DataTypeTok{matchOnGender =} \OtherTok{TRUE}\NormalTok{)}

\NormalTok{subjectsCtc <-}\StringTok{ }\KeywordTok{selectSubjectsToInclude}\NormalTok{(}\DataTypeTok{caseCrossoverData =}\NormalTok{ caseCrossoverData,}
                                       \DataTypeTok{outcomeId =} \DecValTok{1}\NormalTok{,}
                                       \DataTypeTok{firstOutcomeOnly =} \OtherTok{TRUE}\NormalTok{,}
                                       \DataTypeTok{washoutPeriod =} \DecValTok{183}\NormalTok{,}
                                       \DataTypeTok{matchingCriteria =}\NormalTok{ matchingCriteria)}
\end{Highlighting}
\end{Shaded}

The other steps remain the same:

\begin{Shaded}
\begin{Highlighting}[]
\NormalTok{exposureStatusCtc <-}\StringTok{ }\KeywordTok{getExposureStatus}\NormalTok{(}\DataTypeTok{subjects =}\NormalTok{ subjectsCtc,}
                                       \DataTypeTok{caseCrossoverData =}\NormalTok{ caseCrossoverData,}
                                       \DataTypeTok{exposureId =} \DecValTok{1124300}\NormalTok{,}
                                       \DataTypeTok{firstExposureOnly =} \OtherTok{FALSE}\NormalTok{,}
                                       \DataTypeTok{riskWindowStart =} \DecValTok{-30}\NormalTok{,}
                                       \DataTypeTok{riskWindowEnd =} \DecValTok{0}\NormalTok{,}
                                       \DataTypeTok{controlWindowOffsets =} \KeywordTok{c}\NormalTok{(}\OperatorTok{-}\DecValTok{60}\NormalTok{))}

\NormalTok{fitCtc <-}\StringTok{ }\KeywordTok{fitCaseCrossoverModel}\NormalTok{(exposureStatusCtc)}

\KeywordTok{summary}\NormalTok{(fitCtc)}
\end{Highlighting}
\end{Shaded}

\begin{verbatim}
#> Case-Crossover fitted model
#> Status: OK
#> 
#>           Estimate lower .95 upper .95   logRr seLogRr
#> treatment  1.31394   0.77897   2.21630 0.27303  0.2667
#> 
#> Counts
#>       Cases Controls Control win. (cases) Control win. (controls)
#> Count  7555     7555                 7555                    7555
#>       Exposed case win. (cases) Exposed control win. (cases)
#> Count                       160                          156
#>       Exposed case win. (controls) Exposed control win. (controls)
#> Count                          136                             147
\end{verbatim}

\hypertarget{acknowledgments}{%
\section{Acknowledgments}\label{acknowledgments}}

Considerable work has been dedicated to provide the
\texttt{CaseCrossover} package.

\begin{Shaded}
\begin{Highlighting}[]
\KeywordTok{citation}\NormalTok{(}\StringTok{"CaseCrossover"}\NormalTok{)}
\end{Highlighting}
\end{Shaded}

\begin{verbatim}
#> 
#> To cite package 'CaseCrossover' in publications use:
#> 
#>   Martijn Schuemie (2018). CaseCrossover: Case-Crossover. R
#>   package version 1.1.0. https://github.com/OHDSI/CaseCrossover
#> 
#> A BibTeX entry for LaTeX users is
#> 
#>   @Manual{,
#>     title = {CaseCrossover: Case-Crossover},
#>     author = {Martijn Schuemie},
#>     year = {2018},
#>     note = {R package version 1.1.0},
#>     url = {https://github.com/OHDSI/CaseCrossover},
#>   }
\end{verbatim}

Furthermore, \texttt{CaseCrossover} makes extensive use of the
\texttt{Cyclops} package.

\begin{Shaded}
\begin{Highlighting}[]
\KeywordTok{citation}\NormalTok{(}\StringTok{"Cyclops"}\NormalTok{)}
\end{Highlighting}
\end{Shaded}

\begin{verbatim}
#> 
#> To cite Cyclops in publications use:
#> 
#> Suchard MA, Simpson SE, Zorych I, Ryan P, Madigan D (2013).
#> "Massive parallelization of serial inference algorithms for
#> complex generalized linear models." _ACM Transactions on Modeling
#> and Computer Simulation_, *23*, 10. <URL:
#> http://dl.acm.org/citation.cfm?id=2414791>.
#> 
#> A BibTeX entry for LaTeX users is
#> 
#>   @Article{,
#>     author = {M. A. Suchard and S. E. Simpson and I. Zorych and P. Ryan and D. Madigan},
#>     title = {Massive parallelization of serial inference algorithms for complex generalized linear models},
#>     journal = {ACM Transactions on Modeling and Computer Simulation},
#>     volume = {23},
#>     pages = {10},
#>     year = {2013},
#>     url = {http://dl.acm.org/citation.cfm?id=2414791},
#>   }
\end{verbatim}


\end{document}
